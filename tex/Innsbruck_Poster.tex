\documentclass[25pt, a1paper, portrait]{tikzposter}

\title{Regional Industrialisation in (Southwest) Germany: \hspace Patterns, Diffusion and Long-Term Effects}
\author{Sebastian T. Braun, Richard Franke, Timur Öztürk*}
\date{\today}
\institute{University of Bayreuth}

\usepackage{blindtext}
\usepackage{comment}

\usetheme{Envelope}

\begin{document}

\maketitle

\block{~}
{
    \blindtext
}

\begin{columns}
    \column{0.4}
    \block{More text}{Germany’s rapid industrialization in the mid and late 19th century revolutionized the country technologically, economically, and socially (e.g. Ziegler, 2012). New technologies and forms of production spread, novel industries emerged and modern economic growth began. As in other countries, industrialization was accompanied by sharply falling mortality and fertility rates (e.g. Knodel, 1974; Brown and Guinnane, 2018). Importantly, these transition processes occurred unevenly across the country (Frank, 1994; Kiesewetter, 2004). Documenting regional heterogeneity is thus essential for understanding the Industrial Revolution. }
    
    \column{0.6}
    \block{Something else}{Our research aims to shed new light on three core – but hitherto under-researched aspects of regional industrialization in Germany: (1) the diffusion of steam technology across plants, (2) the micro-geography of the demographic transition at the parish level and its economic causes, and (3) the long-term growth consequences of regional industrialization trajectories. With this aim, the project will compile – and make publicly available – three novel datasets that will significantly enrich the sources available for quantitative studies on regional industrialization in Germany. All three datasets provide information at a significantly more disaggregated regional level than existing sources.}
    \note[
        targetoffsetx=-9cm, 
        targetoffsety=-6.5cm, 
        width=0.5\linewidth
        ]
        {e-mail \texttt{timur.oeztuerk@uni-bayreuth.de}}
\end{columns}

\begin{columns}
    \column{0.5}
    \block{A figure}
    {
        \begin{tikzfigure}
            
        \end{tikzfigure}
    }
    \column{0.5}
    \block{Description of the figure}{\blindtext}
\end{columns}

\end{document}